\documentclass{tufte-handout}
\title{Inpatient Internal Medicine Guidelines}
\author{Andrew Zimolzak, MD, MMSc}
\date{May 5, 2023}

\begin{document}

\maketitle

\section{Expectations}

Be conscientious. Ask if you don't know something. \emph{Contact me}
at any time.\footnote{Don't use personal devices to text medical
information. Don't use personal devices to send photos of ECGs or
rashes, even if no identifiers, per Dr.\ Catic, ACOS Education, March
2023.} I \emph{like} being called about lots of things. Feel free to
call me for any patient status changes, even if minor. Never worry,
`Is this a big enough status change to call him?' If you are ever
wondering \emph{whether} you should call, then please call.

Don't criticize or be negative about others\footnote{Meaning
practically anyone: consultants, technicians, nurses, this hospital,
``outside hospital,'' \emph{etc.} Actual things I have heard in
hallways at other hospitals: ``I don't want this patient and am
attempting to block.'' And: ``We can't get an ABG back quickly because
this is a ghetto hospital.''} while ``on stage'' in public, but it is
OK to vent when we're ``backstage'' in the work room or somewhere at
least as private.

The \emph{ideal resident} is thinking ahead and has a plan. As
resident, you should run the team. Keep us efficient. Be highly
available to your interns (touch base a few times daily; more if
needed). Review their orders. Give feedback. Balance the workload. The
\emph{ideal intern} is organized and diligent. Help your students. You
are the default writer of orders and communicator (with your patients,
with consultants) but the resident will balance this work as needed.
The \emph{ideal medical student} is inquisitive. Give the presentation
every day on your own patients. Learn the basics and tell us what you
learned.

Let me know if a patient makes you feel unsafe or uncomfortable. This
can take many forms.\footnote{Such as casually bigoted, intentionally
insulting, or objectifying comments. Further reading: Jain SH. The
racist patient. \emph{Ann Intern Med.} 2013 Apr 16;158(8):632. PMID:
23588752.} We are here to learn, but not at any cost.



\section{Recommendations}

\paragraph{Oral presentations:} On new patients, focus on the chief
complaint. It's OK to use a short ``with a history
of\ldots{}''\ phrase in your one-liner, but keep chief complaint
\emph{near} the beginning. On existing patients, often there is one
diagnosis that explains why they are still in the hospital. This one
diagnosis should be the bulk of your one-liner. No need to repeat
anything on existing patient oral presentations.\footnote{That means
Assessment should simply be ``Problem 1, Hypercalcemia due
to\ldots{}''\ rather than repeating ``85 year old with a history of
stage IV non small cell lung cancer status post chemotherapy with
cisplatin and etoposide, initiated in January\ldots{}, also with COPD,
CAD, chronic kidney disease\ldots{}.''}

\paragraph{Written documentation:} Work hard on including an accurate chief
complaint on initial H\&P notes. Good questions to ask of patients:
``What do you do for work?'' ``Who lives with you at home?'' If the
patient is a transfer, know where he/she transferred from. Progress
notes can be concise: no need to mention labs in detail, unless it
helps you. For a very short admission, it is fine to have a very short
discharge summary. I prefer to give feedback on written documentation
using a paper copy.

\paragraph{Teaching/EBM:} Be ready for rounds right on time.
Team members may give a short ($\sim$ 10 minute) topic presentation. I
aim to do these first thing if possible. The topic will probably
relate to a question that came up on rounds. I will keep a watch for
topics that fit this bill, but chime in if you want to present about a
specific topic. If possible, make a very short (0.5--1 page, or <300
words) printout for us to read. Spend 30--60 minutes preparing this.
We will try to make it a very directed question so it is answerable
with this amount of effort.\footnote{Specialized things I can present
include: diabetes management for inpatient medicine, diagnostic error,
using medical record data in your research, medication adherence,
prediction models, topics in machine learning, and interpretation of
statistics.} We may also have very short questions that people look up
and present the next day (no printout).




\paragraph{Daily routine:} Since the pandemic, I usually ``table
round'' in the morning. Then I'll ``walk round'' with/without house
staff on patients who are new, sick, or who have status changes,
questions, or bedside teaching points. I don't need to see patients
the morning before discharge. Let me know if workload prevents you
from attending conferences. I usually ``card flip'' quickly with the
resident only, in the afternoon. Afternoons sometimes involve topic
presentations, radiology image review, observed student H\&Ps, or
feedback. On call days, we can staff patients the same day, or discuss
by phone, or wait until morning rounds, based on the resident's
judgment. We can change any aspect of rounds (\emph{e.g.}, the amount
of bedside rounds) based on learning or work needs.

There are no ``standard AM labs.'' Please look for reasons to
\emph{stop} telemetry and daily CBC or BMP rather than reasons to
continue. This is especially true for patients who are ``placement.''




\section{About me}

I grew up in Michigan. My undergraduate degree is in biochemistry from
Michigan State. Medical school was at Washington University in
St.\ Louis. Intern year was at University of Missouri-Columbia, and
residency was at Saint Louis University, with an extra year for chief
residency. I've worked in VAs since medical school. Attending roles
have been mainly VA inpatient and non-VA urgent care.

\marginnote{Document copyright 2018--2023, Andrew Zimolzak. You are
  free to share/adapt under CC BY-NC-SA 4.0. See
  github.com/zimolzak/inpatient-guidelines}

About 75\% of my time is research, specifically \emph{clinical
research informatics,} which means ``using clinical and data warehouse
expertise to make data repositories useful for others' studies.'' I
completed a master's degree in biomedical informatics at Harvard
Medical School and worked at VA Boston from 2014--2018. In Houston, I
work at the IQuESt health services research center, developing
electronic methods to detect and reduce diagnostic error; and at the
clinical/translational institute, promoting use of BCM clinical data
for research.

\end{document}

% LocalWords:  Hypercalcemia CPRS Ps VAs statins MAVERIC genotyping phenotyping
% LocalWords:  HCTZ cisplatin etoposide ABG Catic
