\documentclass{tufte-handout}
\title{Inpatient Internal Medicine Guidelines}
\author{Andrew Zimolzak, MD, MMSc}
\date{May 27, 2022}

\begin{document}

\maketitle

\section{Expectations}

Be conscientious. Ask if you don't know something. \emph{Call me} at
any time or page me.\footnote{E-mail
Andrew.Zimolzak@va.gov for non-urgent things. No patient identifiers
by text message.} I \emph{like} being called about lots of things.
Feel free to call me for any patient status changes, even if minor.
Never worry, `Is this a big enough status change to call him?' If you
are ever wondering \emph{whether} you should call/page, then please
call/page. Don't criticize or be negative about
others\footnote{Meaning practically anyone: consultants, technicians,
nurses, this hospital, ``outside hospital,'' etc. Actual things I have
heard in hallways at other hospitals: ``I don't want this patient and
am attempting to block.'' And: ``We can't get an ABG back quickly
because this is a ghetto hospital.''} while ``on stage'' in public,
but it is OK to vent when we're ``backstage'' in the work room or
somewhere at least as private.

The \emph{ideal resident} is thinking ahead and has a plan. As
resident, you should run the team. Keep us efficient. Be highly
available to your interns (touch base a few times daily; more if
needed). Review their orders. Give feedback. Balance the workload. The
\emph{ideal intern} is organized and diligent. Help your students. You
are the default writer of orders and communicator (with your patients,
with consultants) but the resident will balance this work as needed.
The \emph{ideal medical student} is inquisitive. Give the presentation
every day on your own patients. Learn the basics and tell us what you
learned.




\section{Recommendations}

\marginnote{Everything from here down is ``recommended'' or ``nice to
  have'' but not ``required.''}

\paragraph{Oral presentations:} On new patients, focus on the chief
complaint. It's OK to use a short ``with a history of\ldots{}'' phrase
in your one-liner, but keep chief complaint \emph{near} the beginning
of this sentence. On existing patients, often there is one diagnosis
that explains why he/she is still in the hospital. This one diagnosis
should be the bulk of your one-liner. No need to repeat anything on
existing patient oral presentations.\footnote{That means Assessment
should simply be ``Problem 1, Hypercalcemia due to\ldots{}'' rather
than repeating ``85 year old with a history of stage IV non small cell
lung cancer status post chemotherapy with cisplatin and etoposide,
initiated on\ldots{}, also with COPD, CAD, chronic kidney
disease\ldots{}.''}

\paragraph{Written documentation:} Put the chief complaint in exact
quotes on initial H\&P write-ups. Good questions to ask of patients:
``Who's your primary doctor?'' ``What do you do for work?'' ``Who
lives with you at home?'' I encourage calling or otherwise notifying
the PCP of their patients' admission. If the patient is a transfer,
know where he/she transferred from. Progress notes can be very short.
There is no need to mention labs in detail, unless it helps you review
them or something similar. There are no ``standard AM labs;'' not
every patient needs every lab every morning. This is especially true
for patients who are ``placement.'' For a very short admission, it is
completely fine to have a very short discharge summary. I tend to give
more feedback on oral presentations, but I'm happy to critique written
documentation if you hand me a printout (best for PHI reasons).

\paragraph{Initial 10 minute topic:} Be ready for rounds right on time.
Team members may give a short topic presentation, as in 10 minutes. I
aim to do these first thing if possible. The topic will probably come
from a question that comes up on rounds. I will keep a watch for
topics that fit this bill, but chime in if you want to present about a
specific topic. Make a very short (0.5--1 page, or <300 words)
printout for us to read if at all possible. Spend 30--60 minutes
preparing this. We will try to make it a very directed question so it
is answerable with this amount of effort.\footnote{Specialized things
I can present include use of VA data for research, diagnostic error,
medication adherence, interpretation of papers especially prediction
models or machine learning, biostatistics and epidemiology, diabetes
management for inpatient medicine, and CPRS tips and tricks.} We may
also have very short questions that people look up and present the
next day (no printout).




\paragraph{Rounds:} In the pandemic era, we usually ``table
round'' after morning report. Then we'll see any patients who are
sick, who have something I need to see them for, or who have bedside
teaching points. I don't need to see patients the morning before
discharge. Let me know if workload prevents you from attending
conferences. I usually ``card flip'' quickly with the resident only,
in the afternoon. Afternoons sometimes involve topic presentations,
radiology image review, observed student H\&Ps, or feedback. On call
days, we can staff patients the same day, or discuss by phone, or wait
until morning rounds, based on the resident's judgment.




\paragraph{Safety:} Let me know if a patient makes you feel unsafe.
This can take the form of casually bigoted or objectifying comments,
or other behaviors. We are here to learn, but not at any cost.
\footnote{Further reading: Jain SH. The racist patient. \emph{Ann
Intern Med.} 2013 Apr 16;158(8):632. PMID: 23588752.}




\section{About me}

I grew up in Michigan. My undergraduate degree is in biochemistry from
Michigan State. Medical school was at Washington University in
St.\ Louis. Intern year was at University of Missouri-Columbia, and
residency was at Saint Louis University, with an extra year for chief
residency. I've worked in VAs since medical school.

\marginnote{Document copyright 2018-2022, Andrew Zimolzak. You are
  free to share/adapt under CC BY-NC-SA 4.0. See
  github.com/zimolzak/inpatient-guidelines}

I completed a master's degree in biomedical informatics at Harvard
Medical School. I worked at VA Boston from 2014 to 2018, doing
\emph{clinical research informatics,} which means ``using clinical and
data warehouse expertise to make VA data usable for
research.'' I moved to Houston in 2018. My projects here
include a trial within the VA of an intervention to improve follow-up
on clinical test results.

\end{document}

% LocalWords:  Hypercalcemia CPRS Ps VAs statins MAVERIC genotyping phenotyping
% LocalWords:  HCTZ cisplatin etoposide ABG
